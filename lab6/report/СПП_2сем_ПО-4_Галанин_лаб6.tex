\documentclass[12pt, a4paper, simple]{eskdtext}

\usepackage{hyperref}
\usepackage{_env/gpi_global.env}
\usepackage{_env/gpi_report.env}
\usepackage{_sty/gpi_lst}
\usepackage{_sty/gpi_toc}
\usepackage{_sty/gpi_t}
\usepackage{_sty/gpi_p}
\usepackage{_sty/gpi_u}

% Код
% \ESKDletter{О}{Л}{Р}
% \def \gpiDocTypeNum {81}
% \def \gpiDocVer {00}
% \def \gpiCode {\ESKDtheLetterI\ESKDtheLetterII\ESKDtheLetterIII.\gpiStudentGroupName\gpiStudentGroupNum.\gpiStudentCard-0\gpiDocNum~\gpiDocTypeNum~\gpiDocVer}

\def \gpiDocTopic {Отчёт лабораторной работы №\gpiDocNum}

% Графа 1 (наименование изделия/документа)
% \ESKDcolumnI {\ESKDfontII \gpiTopic \\ \gpiDocTopic}

% Графа 2 (обозначение документа)
% \ESKDsignature {\gpiCode}

% Графа 9 (наименование или различительный индекс предприятия) задает команда
% \ESKDcolumnIX {\gpiDepartment}

% Графа 11 (фамилии лиц, подписывающих документ) задают команды
% \ESKDcolumnXIfI {\gpiStudentSurname}
% \ESKDcolumnXIfII {\gpiTeacherSurname}
% \ESKDcolumnXIfV {\gpiTeacherSurname}

\begin{document}
    \begin{ESKDtitlePage}
    \ESKDstyle{empty}
    \begin{center}
        \gpiMinEdu \\
        \gpiEdu \\
        \gpiKaf \\
    \end{center}

    \vfill

    \begin{center}
        \gpiTopic
    \end{center}

    \vfill

    \begin{center}
        \textbf{\gpiDocTopic} \\
        ПО ДИСЦИПЛИНЕ \gpiDiscipline \\
    \end{center}

    \vfill

    \begin{flushright}
        \begin{minipage}[t]{7cm}
            Выполнил:\\
            \PageTitleStudentInfo
            \PageTitleDateField
            \hspace{0pt}

            Проверил:\\
            \PageTitleTeacherInfo
            \PageTitleDateField
        \end{minipage}
    \end{flushright}

    \vfill

    \begin{center}
        \PageTitleCity~\ESKDtheYear
    \end{center}
\end{ESKDtitlePage}

    \ESKDstyle{empty}
    \begin{center}
        \textbf{\gpiDocTopic}
    \end{center}

    % = = = = = = = =
    \paragraph{} \textbf{Тема}: <<\gpiTopicRep>>

    \paragraph{} \textbf{Цель}: освоить приемы разработки оконных клиент-серверных приложений на Java с использованием сокетов.

    \paragraph{} \textbf{Что нужно сделать}:

    Разработать клиент-серверное оконное приложение на Java с использованием сокетов и JavaFX.

    Можно сделать одну программу с сочетанием функций клиента и сервера либо две отдельных
    (клиентская часть и серверная часть).
    Продемонстрировать работу разработанной программы в сети, либо локально (127.0.0.1).
    Лабораторную работу разрешается выполнять в команде из 2-х человек.

    \begin{center}
        \textbf{Вариант 8}
    \end{center}

    Игра <<Крестики-нолики>>. Классическая игра для двух игроков на поле 3х3.

    % \paragraph{} \textbf{Разработка дизайна}:

    % \begin{figure}[!h]
    %     \centering
    %     \includegraphics[]
    %         {_assets/ClassDiagram.png}
    %     \caption{Диграмма классов}
    % \end{figure}

    \paragraph{} \textbf{Исходный код}: 

    \lstinputlisting[language=java, name=src/com/.../TicTacToe.java]
    {../sources/TicTacToe/src/com/mrwayfarout/tictactoe/TicTacToe.java}

    % \paragraph{}\textbf{Исходники клиентского приложения:}

    % \begin{enumerate}
    %     \item Main.java - основной класс вызывает графический класс MainFX
    %     \item MainFX.java - вызывает графическое окно MainController
    %     \item MainController.java - основное графическое окно
    %     \item TicTacToeGameClass.java - класс с логикой игры
    %     \item SinglePlayerController.java - окно игры с ботов
    %     \item MultiPlayerController.java - окно сетевой игры
    %     \item main-view.fxml - дизайн основного окна
    %     \item single-player-view.fxml - дизайн игры с ботом
    %     \item multi-player-view.fxml - дизайн сетевой игры
    % \end{enumerate}

    % \paragraph{}\textbf{Исходники серверного приложения:}

    % \begin{enumerate}
    %     \item Main.java - основной класс с логикой сервера
    % \end{enumerate}

    % \paragraph{}\textbf{Исходники клиентского приложения:}

    % \lstinputlisting[language=java, name=src/main/java/com/example/tictactoeclientgui/Main.java]
    % {../sources/TicTacToeClientGUI/src/main/java/com/example/tictactoeclientgui/Main.java}

    % \lstinputlisting[language=java, name=src/main/java/com/example/tictactoeclientgui/MainFX.java]
    % {../sources/TicTacToeClientGUI/src/main/java/com/example/tictactoeclientgui/MainFX.java}

    % \lstinputlisting[language=java, name=src/main/java/com/example/tictactoeclientgui/MainController.java]
    % {../sources/TicTacToeClientGUI/src/main/java/com/example/tictactoeclientgui/MainController.java}

    % \lstinputlisting[language=java, name=src/main/java/com/example/tictactoeclientgui/TicTacToeGameClass.java]
    % {../sources/TicTacToeClientGUI/src/main/java/com/example/tictactoeclientgui/TicTacToeGameClass.java}

    % \lstinputlisting[language=java, name=src/main/java/com/example/tictactoeclientgui/SinglePlayerController.java]
    % {../sources/TicTacToeClientGUI/src/main/java/com/example/tictactoeclientgui/SinglePlayerController.java}

    % \lstinputlisting[language=java, name=src/main/java/com/example/tictactoeclientgui/MultiPlayerController.java]
    % {../sources/TicTacToeClientGUI/src/main/java/com/example/tictactoeclientgui/MultiPlayerController.java}

    % \lstinputlisting[language=xml, name=src/main/resources/com/example/tictactoeclientgui/main-view.fxml]
    % {../sources/TicTacToeClientGUI/src/main/resources/com/example/tictactoeclientgui/main-view.fxml}

    % \lstinputlisting[language=xml, name=src/main/resources/com/example/tictactoeclientgui/single-player-view.fxml]
    % {../sources/TicTacToeClientGUI/src/main/resources/com/example/tictactoeclientgui/single-player-view.fxml}

    % \lstinputlisting[language=xml, name=src/main/resources/com/example/tictactoeclientgui/multi-player-view.fxml]
    % {../sources/TicTacToeClientGUI/src/main/resources/com/example/tictactoeclientgui/multi-player-view.fxml}

    % \paragraph{}\textbf{Исходники серверного приложения:}

    % \lstinputlisting[language=java, name=src/com/company/Main.java]
    % {../sources/TicTacToeServer/src/com/company/Main.java}

%     \begin{lstlisting}[caption=Вывод в консоль]
%  Hello, World!
% \end{lstlisting}

    % \paragraph{} \textbf{Вывод}:

    % Создали дизайн используя JavaFX через приложение Scene Builder.
    % Создавали многооконные придложения используя класс Stage и загружая дизайн из файла *.fxml используя класс FXMLLoader.
    
    % Создали сокет-сервер, которые принимает сообщения.

    % В клиенском приложении создали однопользовательскую игру <<Крестики-нолики>> с ботом.

    % В клиенском приложении организовали подключение к серверу.

    % В клиентском приложении организовали соединение сокета и отправку сообщений серверу.
    
    % = = = = = = = =
    \newpage
    % \addcontentsline{toc}{section}{Список использованных источников}
    % \section*{Список использованных источников}
    \paragraph{} \textbf{Список использованных источников}:
    \begin{enumerate}
        \item[1.] Learn Socket Programming in Java with Example in Hindi - YouTube -- [Электронный ресурс]
        - Режим доступа: \url{https://www.youtube.com/watch?v=wndMuud3AT8}.
        Дата~доступа:~01.05.2022.
        \item[2.] Классы Socket и ServerSocket в Java -- [Электронный ресурс]
        - Режим доступа: \url{https://javarush.ru/groups/posts/654-klassih-socket-i-serversocket-ili-allo-server-tih-menja-slihshishjh}.
        Дата~доступа:~01.05.2022.
        \item[3.] Как создать исполняемый jar файл в IntelliJ IDEA - YouTube -- [Электронный ресурс]
        - Режим доступа: \url{https://www.youtube.com/watch?v=tA8rEz_xFrQ}.
        Дата~доступа:~01.05.2022.
        \item[4.] Setup IntelliJ IDEA (2021) for JavaFX \& SceneBuilder and Create Your First JavaFX Application - YouTube -- [Электронный ресурс]
        - Режим доступа: \url{https://www.youtube.com/watch?v=ZfaPMLdgJxQ}.
        Дата~доступа:~01.05.2022.
        \item[5.] JavaFX - Opening an FXML file in New Window - YouTube -- [Электронный ресурс]
        - Режим доступа: \url{https://www.youtube.com/watch?v=ZzwvQ6pa_tk}.
        Дата~доступа:~01.05.2022.
        \item[6.] How To Fix JavaFX runtime components are missing and are required to run this application - YouTube -- [Электронный ресурс]
        - Режим доступа: \url{https://www.youtube.com/watch?v=sdkW_cUH3hw}.
        Дата~доступа:~01.05.2022.
        \item[7.] Export JavaFX 11, 15 or 17 projects into an executable jar file with IntelliJ [2022] - YouTube -- [Электронный ресурс]
        - Режим доступа: \url{https://www.youtube.com/watch?v=F8ahBtXkQzU}.
        Дата~доступа:~01.05.2022.
        \item[8.] How to make Tic Tac Toe game using JavaFX | Java Game Development - YouTube -- [Электронный ресурс]
        - Режим доступа: \url{https://www.youtube.com/watch?v=tZlZ04Sy3uc}.
        Дата~доступа:~21.05.2022.
    \end{enumerate}
    \newpage
\end{document}
