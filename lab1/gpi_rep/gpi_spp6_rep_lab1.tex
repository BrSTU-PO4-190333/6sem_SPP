\documentclass[12pt, a4paper, simple]{eskdtext}

\usepackage{hyperref}
\usepackage{env}
\usepackage{_sty/gpi_lst}
\usepackage{_sty/gpi_toc}
\usepackage{_sty/gpi_t}
\usepackage{_sty/gpi_p}
\usepackage{_sty/gpi_u}

% Код
\ESKDletter{О}{Л}{Р}
\def \gpiDocTypeNum {81}
\def \gpiDocVer {00}
\def \gpiCode {\ESKDtheLetterI\ESKDtheLetterII\ESKDtheLetterIII.\gpiStudentGroupName\gpiStudentGroupNum.\gpiStudentCard-0\gpiDocNum~\gpiDocTypeNum~\gpiDocVer}

\def \gpiDocTopic {ОТЧЁТ ЛАБОРАТОРНОЙ РАБОТЫ}

% Графа 1 (наименование изделия/документа)
\ESKDcolumnI {\ESKDfontII \gpiTopic \\ \gpiDocTopic}

% Графа 2 (обозначение документа)
\ESKDsignature {\gpiCode}

% Графа 9 (наименование или различительный индекс предприятия) задает команда
\ESKDcolumnIX {\gpiDepartment}

% Графа 11 (фамилии лиц, подписывающих документ) задают команды
\ESKDcolumnXIfI {\gpiStudentSurname}
\ESKDcolumnXIfII {\gpiTeacherSurname}
\ESKDcolumnXIfV {\gpiTeacherSurname}

\begin{document}
    \input{_tex/gpi_rep_titlePage.tex}
    \ESKDstyle{empty}
    
    %
    \paragraph{}\textbf{Задание}
    
    Разработать консольное приложение с двумя потоками (1 - основной поток; 2 - поток, в котором производится вычисление).

    \begin{center}
        \textbf{Вариант 5}
    \end{center}

    $$\sum_{k=1}^n {1 \over (2k-1) (2k + 1)} = {1 \over 1 * 3} + {1 \over 3 * 5} + ... + {1 \over (2n-1) (2n + 1)}$$

    \paragraph{}\textbf{Исходный код}

    \lstinputlisting[language=java, name=Main.java]
        {../gpi_src/src/io/github/Pavel_Innokentevich_Galanin/Main.java}
    \lstinputlisting[language=java, name=Spp6Lab1Option5_Thread.java]
        {../gpi_src/src/io/github/Pavel_Innokentevich_Galanin/Spp6Lab1Option5_Thread.java}

    \paragraph{}\textbf{Результат программы}

\begin{lstlisting}[name=Результат программы]
Основной поток создан
Дочерний поток создан
n = 30000
sum = 0.49998402860409225
Дочерний поток закончен
Основной поток закончен
\end{lstlisting}

    \paragraph{}\textbf{Вывод}: Реализовали класс - поток (наследовали Thread).
    Реализовал вычисление в отдельном потоке (переопределили метод run, а вызывали, используя метод start).
    Присоединил поток к основному потоку, используя метод join, для того, чтобы основной поток не завершался, пока не выполнится дочерний.

    %
    \section*{СПИСОК ИСПОЛЬЗОВАННЫХ ИСТОЧНИКОВ}
    \addcontentsline{toc}{section}{СПИСОК ИСПОЛЬЗОВАННЫХ ИСТОЧНИКОВ}
    \begin{enumerate}
        \item[1.] Thread в Java: Часть I — потоки - [Электронный ресурс]
        URL: \url{https://javarush.ru/groups/posts/2047-threadom-java-ne-isportishjh--chastjh-i---potoki}
        (дата~обращения:~15.02.2022).
        \item[2.] Метод Thread.join() - [Электронный ресурс]
        URL: \url{https://javarush.ru/groups/posts/1993-mnogopotochnostjh-chto-delajut-metodih-klassa-thread}
        (дата~обращения:~15.02.2022).
    \end{enumerate}
    \newpage
\end{document}
