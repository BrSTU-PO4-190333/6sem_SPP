\documentclass[12pt, a4paper, simple]{eskdtext}

\usepackage{hyperref}
\usepackage{_env/gpi_global.env}
\usepackage{_env/gpi_report.env}
\usepackage{_sty/gpi_lst}
\usepackage{_sty/gpi_toc}
\usepackage{_sty/gpi_t}
\usepackage{_sty/gpi_p}
\usepackage{_sty/gpi_u}

% Код
% \ESKDletter{О}{Л}{Р}
% \def \gpiDocTypeNum {81}
% \def \gpiDocVer {00}
% \def \gpiCode {\ESKDtheLetterI\ESKDtheLetterII\ESKDtheLetterIII.\gpiStudentGroupName\gpiStudentGroupNum.\gpiStudentCard-0\gpiDocNum~\gpiDocTypeNum~\gpiDocVer}

\def \gpiDocTopic {Отчёт лабораторной работы №\gpiDocNum}

% Графа 1 (наименование изделия/документа)
% \ESKDcolumnI {\ESKDfontII \gpiTopic \\ \gpiDocTopic}

% Графа 2 (обозначение документа)
% \ESKDsignature {\gpiCode}

% Графа 9 (наименование или различительный индекс предприятия) задает команда
% \ESKDcolumnIX {\gpiDepartment}

% Графа 11 (фамилии лиц, подписывающих документ) задают команды
% \ESKDcolumnXIfI {\gpiStudentSurname}
% \ESKDcolumnXIfII {\gpiTeacherSurname}
% \ESKDcolumnXIfV {\gpiTeacherSurname}

\begin{document}
    \begin{ESKDtitlePage}
    \ESKDstyle{empty}
    \begin{center}
        \gpiMinEdu \\
        \gpiEdu \\
        \gpiKaf \\
    \end{center}

    \vfill

    \begin{center}
        \gpiTopic
    \end{center}

    \vfill

    \begin{center}
        \textbf{\gpiDocTopic} \\
        ПО ДИСЦИПЛИНЕ \gpiDiscipline \\
    \end{center}

    \vfill

    \begin{flushright}
        \begin{minipage}[t]{7cm}
            Выполнил:\\
            \PageTitleStudentInfo
            \PageTitleDateField
            \hspace{0pt}

            Проверил:\\
            \PageTitleTeacherInfo
            \PageTitleDateField
        \end{minipage}
    \end{flushright}

    \vfill

    \begin{center}
        \PageTitleCity~\ESKDtheYear
    \end{center}
\end{ESKDtitlePage}

    \ESKDstyle{empty}
    \begin{center}
        \textbf{\gpiDocTopic}
    \end{center}

    % = = = = = = = =
    \paragraph{} \textbf{Тема}: <<\gpiTopicRep>>

    \paragraph{} \textbf{Цель}:
    приобрести навыки написания простого оконного многопоточного приложения с использованием Java API.

    \paragraph{} \textbf{Что нужно сделать}:

    Разработать консольное приложение с двумя потоками (1 - основной поток; 2 - поток, в котором производится вычисление).

    \textbf{Вариант 5}:
    $\sum\limits_{k=1}^n {1 \over (2k-1) (2k + 1)} = {1 \over 1 * 3} + {1 \over 3 * 5} + ... + {1 \over (2n-1) (2n + 1)}$

    % \paragraph{} \textbf{Разработка дизайна}:

    % \begin{figure}[!h]
    %     \centering
    %     \includegraphics[]
    %         {_assets/ClassDiagram.png}
    %     \caption{Диграмма классов}
    % \end{figure}

    \paragraph{} \textbf{Исходный код}: 

    \lstinputlisting[language=java, name=src/com/company/Main.java]
    {../sources/SPP_2sem_PO4_Galanin_lab1/src/com/company/Main.java}

    \lstinputlisting[language=java, name=src/com/company/MyThread.java]
    {../sources/SPP_2sem_PO4_Galanin_lab1/src/com/company/MyThread.java}

    \begin{lstlisting}[caption=Вывод в консоль]
Main thread created
Child thread created
n = 1000
sum = 0.49974987493746836
Child thread finished
Main thread finished
\end{lstlisting}

    \paragraph{} \textbf{Вывод}:
    Реализовали класс - поток (наследовали Thread).
    Реализовал вычисление в отдельном потоке (переопределили метод run, а вызывали, используя метод start).
    Присоединил поток к основному потоку, используя метод join, для того, чтобы основной поток не завершался, пока не выполнится дочерний.

    % = = = = = = = =
    % \newpage
    % \addcontentsline{toc}{section}{Список использованных источников}
    % \section*{Список использованных источников}
    \paragraph{} \textbf{Список использованных источников}:
    \begin{enumerate}
        \item[1.] Thread в Java: Часть I - потоки [Электронный ресурс]
        - Режим доступа: \url{https://javarush.ru/groups/posts/2047-threadom-java-ne-isportishjh--chastjh-i---potoki}.
        Дата~доступа:~15.02.2022.
        \item[2.] Метод Thread.join() [Электронный ресурс]
        - Режим доступа: \url{https://javarush.ru/groups/posts/1993-mnogopotochnostjh-chto-delajut-metodih-klassa-thread}.
        Дата~доступа:~15.02.2022.
        \item[3.] Как создать исполняемый jar файл в IntelliJ IDEA - YouTube [Электронный ресурс]
        - Режим доступа: \url{https://www.youtube.com/watch?v=tA8rEz_xFrQ}.
        Дата~доступа:~01.05.2022.
        \item[4.] Export JavaFX 11, 15 or 17 projects into an executable jar file with IntelliJ [2022] - YouTube [Электронный ресурс]
        - Режим доступа: \url{https://www.youtube.com/watch?v=F8ahBtXkQzU}.
        Дата~доступа:~01.05.2022.
    \end{enumerate}
    \newpage
\end{document}
